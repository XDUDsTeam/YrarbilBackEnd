




% 后端文档的主文件

%%%%%%%%%%%
% 宏定义
%%%%%%%%%%%

%flag 设置
\makeatletter
\def\@NoStyleChapter{} % 设置不适用章节格式
\def\@BackendDoc{}
\def\@UsingAppendix{}
\makeatother
%导入导言文件
% 设置格式

%防止重复引用

\ifx \incpreamble \undefined
\def\incpreamble{}
\else
\endinput
\fi

\makeatletter
\def\doclass{
\ifx \@NoCTEX \undefined
\documentclass[UTF8]{ctexart}
\else
\documentclass[UTF8]{article}
\usepackage{xeCJK}
\fi
}
\doclass
\makeatother

\makeatletter
\ifx \@NoLiterateHaskell \undefined
\usepackage{xltxtra} %提供 XeLaTeX logo
\usepackage{amsmath}
\usepackage{listings}
\usepackage{color,xcolor}
\newcounter{codeline}
\setcounter{codeline}{1}
\lstnewenvironment{code}{firstnumber=last,language=Haskell,breaklines,backgroundcolor=\color[rgb]{1.00,0.90,1.00},basicstyle=\sffamily,keywordstyle=\bfseries\color[rgb]{0.16,0.53,0.53},commentstyle=\rmfamily\itshape,flexiblecolumns,numbers=left,numberstyle=\tiny,frame=trBL,label=sourceCtr}}{\setcounter{codeline}{\value{lstnumber}}}
\lstnewenvironment{spec}{language=Haskell,breaklines,backgroundcolor=\color[rgb]{0.36,0.67,0.78},basicstyle=\sffamily,keywordstyle=\bfseries\color[rgb]{0.16,0.53,0.53},commentstyle=\rmfamily\itshape,flexiblecolumns,numbers=left,numberstyle=\tiny,frame=trBL,label=sourceCtr}}{\setcounter{lstnumber}{\value{codeline}}}
\lstnewenvironment{example}{language=Haskell,breaklines,backgroundcolor=\color[rgb]{0.98,0.72,0.43},basicstyle=\sffamily,keywordstyle=\bfseries\color[rgb]{0.16,0.53,0.53},commentstyle=\rmfamily\itshape,flexiblecolumns,numbers=left,numberstyle=\tiny,frame=trBL,label=sourceCtr}}{\setcounter{lstnumber}{\value{codeline}}}
\lstnewenvironment{jsonlists}{language=JavaScript,breaklines,basicstyle=\sffamily,keywordstyle=\bfseries\color[rgb]{0.16,0.53,0.53},commentstyle=\rmfamily\itshape,flexiblecolumns,numbers=left,numberstyle=\tiny,frame=trBL,label=sourceCtr}}{\setcounter{lstnumber}{\value{codeline}}}
\long\def\ignore#1{\relax}
\fi
\makeatother

%如果没有 CTEX



\makeatletter
\ifx \@NoJavaScript \undefined
\lstdefinelanguage{JavaScript}
{
  morekeywords={true,false,null},
  alsoletter={:},
  moredelim=[s][{\color[rgb]{0.67,0.00,0.67}}]{"}{"},
  moredelim=[s][{\color[rgb]{0.36,0.67,0.78}}]{:"}{"},
  identifierstyle=\color{blue}
}[keywords,comments,strings]
\fi
\makeatother

\usepackage[colorlinks,linkcolor=blue,anchorcolor=blue,citecolor=red,bookmarksnumbered]{hyperref}

%定义 BibTeX 图标
\def\BibTeX{{\rm B \kern-.05em{\sc i\kern-.025em b}\kern-.08em T\kern-.1667em\lower.7ex\hbox{E}\kern-.125emX}}

\author{Qinka\\qinka@live.com}
\title{Yrarbil 后端参考文档}


\begin{document}
  \maketitle
  \newpage
  % license.tex
  \section*{LICENSE}\label{license}
  \# BSD3 License
  
  \# we try to change this license to yaml formal
  
  \def\qqquad{\quad \qquad}
  Copyright (c) 2015, XDUDsTeam \\
  \hspace*{4em} MEMBER\\
  \hspace*{6em} - name: Qinka\\
  \hspace*{6.75em}email: qinka@live.com\\
  \hspace*{6em} - name: qiyuyi\\
  \hspace*{6.75em}email: 631987611@qq.com\\
  \hspace*{6em} - name: starsriver\\
  \hspace*{6.75em}email: starsriver@outlook.com\\
  \hspace*{6em} - name: woyijkl1\\
  \hspace*{6.75em}email: yinrongdi@163.com\\
  \hspace*{6em} - name: daisycx \\
  \hspace*{6.75em}email: 729227860@qq.com
  \vspace{0.5em}
  
  All rights reserved.
  \vspace*{0.5em}
  
  Redistribution and use in source and binary forms, with or without
  modification, are permitted provided that the following conditions are met:
  \vspace{.5em}
  
  - Redistributions of source code must retain the above copyright notice, this
  list of conditions and the following disclaimer.
  \par \vspace{0.25em}
  - Redistributions in binary form must reproduce the above copyright notice,
  this list of conditions and the following disclaimer in the documentation
  and/or other materials provided with the distribution.
  \par \vspace{0.25em}
  - Neither the name of Yrarbil nor the names of its
  contributors may be used to endorse or promote products derived from
  this software without specific prior written permission.
  
  \vfill
  \textbf{THIS SOFTWARE IS PROVIDED BY THE COPYRIGHT HOLDERS AND CONTRIBUTORS "AS IS"
      AND ANY EXPRESS OR IMPLIED WARRANTIES, INCLUDING, BUT NOT LIMITED TO, THE
      IMPLIED WARRANTIES OF MERCHANTABILITY AND FITNESS FOR A PARTICULAR PURPOSE ARE
      DISCLAIMED. IN NO EVENT SHALL THE COPYRIGHT HOLDER OR CONTRIBUTORS BE LIABLE
      FOR ANY DIRECT, INDIRECT, INCIDENTAL, SPECIAL, EXEMPLARY, OR CONSEQUENTIAL
      DAMAGES (INCLUDING, BUT NOT LIMITED TO, PROCUREMENT OF SUBSTITUTE GOODS OR
      SERVICES; LOSS OF USE, DATA, OR PROFITS; OR BUSINESS INTERRUPTION) HOWEVER
      CAUSED AND ON ANY THEORY OF LIABILITY, WHETHER IN CONTRACT, STRICT LIABILITY,
      OR TORT (INCLUDING NEGLIGENCE OR OTHERWISE) ARISING IN ANY WAY OUT OF THE USE
      OF THIS SOFTWARE, EVEN IF ADVISED OF THE POSSIBILITY OF SUCH DAMAGE.}
  \newpage
  \tableofcontents
  \pdfbookmark[1]{目录}{Qinka}
  \newpage

  \section{说明}
  \subsection{警告}
  运行在32位系统中,可能会产生意料之外的BUG。

  \section[程序主文件 Main.lhs文件]{Main.lhs}
  主文件提供了 程序入口与 Yesod 的基本所需的内容。
  \input{Main.lhs}

  \section[设置载入文件 Main/Config.lhs文件]{Main/Config.lhs}
  提供了对程序设置的读取。
  \input{Main/Config.lhs}

  \section[共用的配置文件 lib/Yrarbil/Backend/Config.lhs]{lib/../Config.lhs}
  用于启动器和主程序的配置的类型文件。
  \input{../lib/Yrarbil/Backend/Config.lhs}

  \section[子站-提供版本信息 Info.lhs文件]{Info.lhs}
  提供了子站-提供版本的信息。
  \input{Info.lhs}

  \section[辅助Info.lhs文件]{Info/Data.lhs}
  \input{Info/Data.lhs}

  \section[认证]{Auth.lhs}
  \input{Auth.lhs}

  \section[辅助Auth.lhs]{Auth/Data.lhs}
  \input{Auth/Data.lhs}

  \section{Launcher:用于道客云}
  用于在道客云等 Docker 容器中启动。
  \input{../launch.docker/Main.lhs}
  \input{../launch.docker/Args.lhs}
  \section{Launcher:通用启动器}
  用于通用环境的启动。
  \input{../launch/Main.lhs}
  \input{../launch/Args.lhs}
\newpage
    \begin{appendices}
        % 参考文档链接
        \def\ApiDocPDF{\href{https://github.com/XDUDsTeam/YrarbilRelease/raw/master/APIDoc.pdf}{API Document}}
        \def\ArchitectureDesignPDF{\href{https://github.com/XDUDsTeam/YrarbilRelease/raw/master/ArchitectureDesign.pdf}{Architecture Design Document}}
        \def\BackEndPdf{\href{https://github.com/XDUDsTeam/YrarbilRelease/raw/master/BackEnd.pdf}{Backend Document}} 
        \def\DemandAnalysisPdf{\href{https://github.com/XDUDsTeam/YrarbilRelease/raw/master/DemandAnalysis.pdf}{Demand Analysis Docment}}
        \def\TestReportPDF{\href{https://github.com/XDUDsTeam/YrarbilRelease/raw/master/TestReport.pdf}{Text Report}}
        
        \makeatletter
        \ifdefined \@NoStyleChapter
        \section{可参考文档}
        \else
        \chapter{可参考文档}
        \fi
        \makeatother
        
        
        Github \& Travis-CI 版
        \begin{itemize}
            \makeatletter
            \ifdefined \@APIDoc
            \relax
            \else
            \item \ApiDocPDF
            \fi
            
            \ifdefined \@ADDoc
            \relax
            \else
            \item \ArchitectureDesignPDF
            \fi
            
            \ifdefined \@BackendDoc
            \relax
            \else
            \item \BackEndPdf
            \fi
            
            \ifdefined \@DADoc
            \relax
            \else
            \item \DemandAnalysisPdf
            \fi
            
            \ifdefined \@TRDoc
            \relax
            \else
            \item \TestReportPDF
            \fi
            \makeatother
        \end{itemize}
    \end{appendices}
\end{document}
